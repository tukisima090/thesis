\documentclass[uplatex,openany,oneside,a4j,11pt]{jsbook}
\usepackage[top=25truemm,bottom=25truemm,left=25truemm,right=25truemm]{geometry}
\usepackage[dvipdfmx]{graphicx}
\usepackage[dvipdfmx]{color}
\usepackage[dvipdfmx,colorlinks,linkcolor=blue,urlcolor=blue,bookmarks=true,bookmarksnumbered=true]{hyperref}
\usepackage{pxjahyper}
\usepackage{bookmark}
\usepackage{url}


%\usepackage[]{biblatex}
%\addbibresource{bibs}

\usepackage{amsmath}
\usepackage{amssymb}
\usepackage{amsfonts}
\usepackage{here}
\usepackage{comment}

%\addbibresource{'C:/Users/miyao/Desktop/MasterThesis/bibs'}
%\usepackage{physics}
\usepackage[style=phys,articletitle=true,biblabel=brackets,chaptertitle=false,pageranges=false,backend=biber]{biblatex}
\addbibresource{bibs.bib}
% \usepackage{amsmath}
% \usepackage{amssymb}
% \usepackage{amsfonts}
% \usepackage{here}
% \usepackage{comment}
%\usepackage{physics}
%\usepackage[backend=biber]{biblatex}
%\usepackage[style=phys]{biblatex}
%\addbibresource{bibs}
\begin{document}
\begin{titlepage}
    \begin{center}
        {\Large 2020 年度 修士論文}\\
        \vspace{180truept}
        {\Huge 超伝導準集sss中定数共振素子における\\
        \vspace{10truept}
        高強度結合の実装}\\ 
        \vspace{70truept}

        {\Large \today}\\

        \vspace{70truept}

        {\Large 東京理科大学大学院 理学研究科物理学専攻 蔡研究室\\
        (学籍番号 1219537)}\\

        \vspace{20truept}

        {\huge 宇井 颯太}\\

        \vspace{160truept}
        {\Large 東京理科大学大学院 理学研究科物理学専攻}\\
    \end{center}
\end{titlepage}
%\maketitle
%\frontmatter
%\addcontentsline{toc}{chapter}{概要}
{\Large 序章}\\

超伝導デバイスを用いて量子状態制御に初めて成功した1999年
昨今様々な手法を用いた量子状態の制御が注目を浴びているが集積力や小型化という面で超伝導を用いたデバイスが最も有力視されている。また、量子デバイスの集積化に伴い、量子情報、や量子科学計算等その応用分野においてもスタートアップやコミュニティ、国家プロジェクトを通じて新たな動きがみられている。\\
こうした時代の中jll、最先端技術を用いで超伝導デバイスの性能向上、集積、応用の場に立てるというのは非常に有意義なことである。この場を借りて教授、先輩方、友人、そして家族に感謝したい。\\
\ さて、超伝nn導デバイス本来の性能向上はもちろん、目的意図に沿ったデバイスの作成が量子コンピューターの実用化という面で必要とされている。ハードとソフトの連携が盛んにおこなわれている中で、情報分野で提唱されたアルゴリズムを実験で実証することは非常に重要なこととなってくる。\\
こうした中で本稿では、産業総合研究所の今福健太郎さんが提唱する、非断熱的な量子アニーリングダイナミクスの新手法についてその物理的現象を実証することを目的としている。\\
構成としては、\\
第1章では、従来hの量子アニssーリングの計算手法について簡潔に説明し、その利点と問題点について解説する。続いて、今福さんの提唱する非断熱量子ダイナミクスについて例を交えながらその意義を解説する。\\
続く第2章ではs、量子ビットについて、とりわけ今回作成した磁束型量子ビットに関して、ジョセフソン接合の物理から解説する。\\
第3章ではjj2量kk子ssssビットのcouplingにつjjjjいて今回採用した方法を解説する。aサンプル作成についてその条件や作成手順について解説する。第5章では実験方法について解説し、第6章で実験結果に言及する。\\
ssss\\
\tableofcontents
%\mainmatter

\chapter{超伝導ss素子}
        2llsssss
        この章で従来の\cite{nakamura1999coherent}量子アニーリング手法について簡便に解説したのち、今福さんによる新手法の意義、方法について述べる。\\
    \section{アニーリング量子計算sss}
基底状態探索として近年量子アニーリングという手法がよく使われている。この計算手法は駆動ハミルトニアンをVと定義したときに系のハミルトニアンを
\begin{equation}
    \hat{H}(t) = s(t) \hat{H} +(1-s(t))\hat{V}
\end{equation}
と定義する。初期状態は$t=0$で$s(t)=0$、$t=t_f$で$s(t_f)=1$であることに注意する。
量子力学の断熱定理によれば系の初期状態を初期ハミルトニアンVで用意すれば逐次的にハミルトニaアン
を変動させることで$t=t_f$ではハミルトニアン$H_f$において基底状態を実現することができるというものである。

上式によればエネルギーが小さければ小さいほ\cite*{Wulschner2016}

\chapter{jyoron}
    2llsssss
    この章で従来sの\cite{nakamura1999coherent}量子アニーリング手法について簡便に解説したのち、今福さんによる新手法の意義、方法について述べる。\\
    \section{アニーリング量子計算sss}
基底状態探索として近年量子アニーリングという手法がよく使われている。この計算手法は駆動ハミルトニアンをVと定義したときに系のハミルトニアンを
\begin{equation}
    \hat{H}(t) = s(t) \hat{H} +(1-s(t))\hat{V}
\end{equation}
と定義する。初期状態は$t=0$で$s(t)=0$、$t=t_f$で$s(t_f)=1$であることに注意する。
量子力学の断熱定理によれば系の初期状態を初期ハミルトニアンVで用意すれば逐次的にハミルトニaアン
を変動させることで$t=t_f$ではハミルトニアン$H_f$において基底状態を実現することができるというものである。

上式によればエネルギーgapが小さければ小さいほ\cite*{Harris2009}

\chapter{jyoron}
    2llsssss
    この章で従来の\cite{nakamura1999coherent}量子アssニーリング手法について簡便に解説したのち、今福さんによる新手法の意義、方法について述べる。\\
    \section{アニーリング量子計算sss}
基底状態探索として近年量子アニーリングという手法がよく使われている。この計算手法は駆動ハミルトニアンをVと定義したときに系のハミルトニアンを
\begin{equation}
    \hat{H}(t) = s(t) \hat{H} +(1-s(t))\hat{V}
\end{equation}
と定義する。初期状態は$t=0$で$s(t)=0$、$t=t_f$で$s(t_f)=1$であることに注意する。
量子力学の断熱定理によれば系の初期状態を初期ハミルトニアンVで用意すれば逐次的にハミルトニaアン
を変動させることで$t=t_f$ではハミルトニアン$H_f$において基底状態を実現することができるというものである。

上式によればエネルギーgapが小さければ小さいほ
\chapter{Appendix}
\begin{abstract}
    ここでdlsssssssssssは、卒業論s文sssssss本旨にjは記載eeeしなかった計算の類をまとめておく。

\end{abstract}
\begin{appendix}
    \section{Appendix}
ここでは、Hamiltonianの変換を行う。
\begin{equation}
    \hat{\mathcal{H}}=\hbar\left(\hat{a}^{\dagger }\ \hat{b}^{\dagger }\right)\left(\begin{array}{cc}
    \tilde{\omega}_{a} & g(\Phi ) \\
    g(\Phi ) & \hat{\omega}_{b}
    \end{array}\right)\left(\begin{array}{l}
    \hat{a} \\
    \hat{b}
    \end{array}\right)
\end{equation}
\begin{equation}
    = \hbar \hat{\omega}_{a} \hat{a}^{\dagger} \hat{a}+\hbar \hat{\omega}_{b} \hat{b}^{\dagger} \hat{b}+\hbar g(\Phi)\left(\hat{a}^{\dagger}\hat{b}+\hat{a} \vec{b}^{+}\right)
\end{equation}
\begin{equation}
    \hat{c}_{\pm}=\frac{\hat{a} \pm \hat{b}}{\sqrt{2}} \quad \hat{c}_{+}^{\dagger}=\frac{\hat{a}^{\dagger} \pm \hat{b}^{\dagger}}{\sqrt{2}}
\end{equation}
\begin{equation}
    \hat{a}^{\dagger}=\frac{\hat{c}_{+}^{\dagger}+\hat{c}_{-}^{\dagger}}{\sqrt{2}} \quad \hat{b}=\frac{\hat{c}_{+}^{\dagger}-\hat{c}_{-}^{\dagger}}{\sqrt{2}}
\end{equation}
\begin{equation}
    \hat{a}=\frac{\hat{c}_{+}+\hat{c}}{\sqrt{2}} \quad \hat{b}=\frac{\hat{c}_{+}-\hat{c}_{-}}{\sqrt{2}}
\end{equation}
\begin{equation}
    \hat{a}^{\dagger} \hat{a}=\frac{1}{2}\left(\hat{c}_{+}^{\dagger}+\hat{c}_{-}^{+}\right)\left(\hat{c}_{t}+\hat{c}_{-}\right) \quad \hat{a}^{\dagger} \hat{b}=\frac{1}{2}\left(\hat{c}_{t}^{+}+\hat{c}_{-}^{+}\right)\left(\hat{c}_{+}-\hat{c}_{-}\right)
\end{equation}
\begin{equation}
    \hat{b} \hat{b}=\frac{1}{2}\left(\hat{c}_{t}+\hat{c}_{-}^{+}\right)\left(\hat{c}_{+}-\hat{c}_{-}\right) \quad \hat{a} \hat{b}^{\dagger}-\frac{1}{2}\left(\hat{c}_{t}+\hat{c}_{-}\right)\left(\hat{c}_{t}^{+}-\hat{c}_{-}^{+}\right)
\end{equation}
\begin{equation}
    \hat{a}^{\dagger} \hat{a}=\frac{1}{2}\left[\hat{c}_{t}^{+} \hat{c}_{+}+\hat{c}_{t}^{+} \hat{c}+\hat{c}_{-}^{+} \hat{c}_{t}+\hat{c}_{-}^{+} \hat{c}_{-}\right]
\end{equation}
\begin{equation}
    \hat{b}^{\dagger} \hat{b}=\frac{1}{2}\left[\hat{c}_{t}^{*} \hat{c}_{t}-\hat{c}_{t}^{n} \hat{c}-\hat{c}_{-}^{+} \hat{c}_{t}+\hat{c}_{-}^{+} \hat{c}-\right]
\end{equation}
\begin{equation}
    \hat{a}^{\dagger} \hat{b}=\frac{1}{2}\left[\dot{c}+\hat{c}_{t}-\hat{c}_{t} \hat{c}_{-}+\hat{c}_{-}^{+} \hat{c}_{t}-\hat{c}_{-}^{\prime} \hat{c}_{-}\right]
\end{equation}
\begin{equation}
    \hat{a} \hat{b}^{\dagger}=\frac{1}{2}\left[\hat{c}_{+} \hat{c}_{+}^{4}-\hat{c}_{+} \hat{c} \pm+\hat{c}-\hat{c}_{+}^{2}-\hat{c}-\hat{c}_{-}\right]
\end{equation}
\begin{equation}
    \hat{H}=\frac{\hbar}{2} \hat{w}_{a}\left[\hat{c}_{t}^{+} \hat{c}_{+}+\hat{c}_{t}^{+} \hat{c}+\hat{c}_{-}^{+} \hat{c}_{t}+\hat{c}_{-}^{+} \hat{c}_{-}\right]
\end{equation}
\begin{equation}
    +\frac{\hbar}{2} \hat{\omega}_{b}\left[\hat{c}_{+}^{+} \hat{c}_{+}-\hat{c}_{+}^{+} \hat{c}_{-}-\hat{c}_{-}^{+} \hat{c}_{+}+\hat{c}_{-}^{+} \hat{c}_{-}\right]
\end{equation}
\begin{equation}
    +\frac{\hbar}{2} g\left[2 \hat{c}+\hat{c}_{+}-2 \hat{c}_{-} \hat{c}\right]
\end{equation}
\begin{equation}
    \hat{H}=\frac{\hbar}{2}\left(\hat{\omega}_{a}+\hat{\omega}_{b}+2 g(\Phi)\right) \hat{c}_{t}^{+} \hat{c}_{+}+\frac{\hbar}{2}\left(\hat{w} a+\hat{w}_{b}-2 g(\Phi)\right) \hat{c}_{-}^{+} \hat{c}_{-}
\end{equation}
\begin{equation}
    +\frac{\hbar}{2} A\left(\hat{w}_{a}-\hat{w}_{b}\right)\left(\hat{c}_{t}+\hat{c}_{-}+\hat{c}_{+} \hat{c}_{-}\right)
\end{equation}
\begin{equation}
    \omega_{a}+\bar{c}_{b}+2 g(\Phi)=\Omega+
\end{equation}
\begin{equation}
    \omega_{a}+\bar{c}_{b}-2 g(\Phi)=\Omega-
\end{equation}
\begin{equation}
    \hat{\omega}_{a}-\hat{w}_{b}=\Delta
\end{equation}
\begin{equation}
    \hat{H}=\frac{\hbar}{2} \Omega+\hat{c}+\hat{c}+\frac{\hbar}{2} \Omega-\hat{c} \pm \hat{c}+\frac{\hbar}{2} \Delta\left(\hat{c}+\hat{c}+\hat{c}_{+} \hat{c}_{-}^{+}\right)
\end{equation}
\begin{equation}
    -\frac{\hbar}{2}\left(\begin{array}{cc}
    \hat{c}_{+} & \hat{c} \pm
    \end{array}\right)\left(\begin{array}{cc}
    \Omega_{1} & \Delta \\
    2 & \Omega_{2}
    \end{array}\right)\left(\begin{array}{l}
    \hat{c}_{+} \\
    \hat{c}_{-}
    \end{array}\right)
\end{equation}
\section{結合素子の諸々計算}
dc-SQUIDのインダクタンスは
\begin{equation}
    L_{s}(\Phi)=\frac{\Phi_0}{4\pi I_{c}|{\cos({\phi_{-}^{min}(\Phi_{ext})}})|}
\end{equation}
と記述することができる。
\begin{equation}
    \Phi=\Phi_{ext}+L_{loop}
\end{equation}
\begin{equation}
    \beta_{dc}=\frac{2\pi L_{loop} I_{c}}{\Phi_{0}}
\end{equation}
とすると。
\section{Fabrication}

\end{appendix}
%\bibliographystyle{}
\printbibliography[title=参考文献]
\end{document}
